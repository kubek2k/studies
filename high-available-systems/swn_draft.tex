\documentclass{article}
\usepackage[latin2]{inputenc}
\usepackage{amsfonts}
\usepackage[MeX]{polski}
\title{Plan przyk�adowego laboratorium z wykorzystaniem Linux-HA 2.0 i DRBD}


\author{Jakub Janczak \& Tomasz Duszka}
\begin{document}
\maketitle
\section{Plan laboratorium}
W oparciu o dystrybucj� Fedora Core - nie b�dzie konieczne tworzenie obraz�w systemu
\subsection{Konwersja pliku /etc/haresources do cib.xml i obserwacja dzia�ania heartbeat 2}
z u�yciem: \\
python /usr/lib/heartbeat/cts/haresources2cib.py > /var/lib/heartbeat/crm/cib.xml \\
uruchomienie klastra w trybie active/passive (z migracja IP i serwera apache)
\subsection{Dodanie monitoringu us�ug i obserwacja dzia�ania klastra}
Killowanie Apache'a etc..

\subsection{Dodanie DRBD jako no�nika stron dla Apache'a}
Obserwacja sytuacji migracji i split-brainu
Przyk�adowe u�ycie programu drbdlinks

\subsection{Ew. U�ycie programu csync jako alternatywy dla rsync}

\section{Ewentualne rozszerzenia}
\subsection{Dodanie STONITH do klastra} - ale to mo�e by� dosy� ci�kie, ale poniek�d spektakularne

\end{document}
