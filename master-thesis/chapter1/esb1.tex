Wstęp teoretyczny

ESB (Enterprise Service Bus) to jedno z podejść które ułatwia implementację architektury SOA. Jest to oparta na standardach platforma integracyjna, która łączy w sobie zalety takich podejść jak komunikacja w oparciu o wiadomości, Web Service'y, transformacje danych, inteligentne wyznaczanie trasy, niezawodność, koordynacja interakcji i transakcyjność w komunikacji pomiędzy różnorodnymi aplikacjami gospodarczymi 7). Cała idea opiera się na istnieniu szyny, za pomocą której odbywa się każda komunikacja w systemie. Dzięki swoim właściwościom szyna jest w stanie sama decydować o tym gdzie ma wysłać daną wiadomość.

tutaj będzie rysunek jak to mniej więcej wygląda
Historia ESB

ESB jako koncepcja nie powstało z przypadku. Na jej kształt miało wpływ wiele doświadczeń z wcześniejszych prób zmierzenia się z problemem integracji usług. Najbardziej znaczący wkład miały koncepcje MOM oraz EAI:
MOM

Idea Message Oriented Middleware 8) powstała w na początku lat 80, jako odpowiedź na potrzeby skomunikowania dynamicznie rozwijających się systemów komupterowych. Opiera się ona na koncepcji wymiany jednostek danych (wiadomości) z pomocą ustalonych protokołów komunikacyjnych. Zazwyczaj komunikacja ta odbywa się asynchronicznie, a jej uczestnicy nie są świadomi istnienia drugiej strony (tj. aplikacja nie jest zainteresowana tym co jest po drugiej stronie kanału komunikacyjnego, po prostu wysyła i/lub odbiera dane).

Tryby komunikacji

MOM w założeniach powinno obsługiwać dwa tryby komunikacji:

    *
	      Send/Receive (wyślij/odbierz) - tryb w którym istnieje tylko jeden producent i jeden konsument wiadomości
		      *
			        Publish/Subscribe (publikuj/zapisz się) - tryb w którym istnieje jeden producent, a odbiorców może być dowolna ilość

					Na początku koncepcja ta realizowana była na poprzez narzucanie własnego standardu komunikacji w obrębie aplikacji, jednak z czasem dopracowano się rozwiązań i standardów które są z powodzeniem użytkowane w wielu aplikacjach. Do tych należą między innymi:

					    *
						      JMS (Java Messaging System) - najszerzej stosowany standard w oparciu o który zostało stworzonych kilka poważnych implementacji stosowanych głównie w serwerach aplikacyjnych JEE takich jak OpenJMS (najszerzej stosowany), OpenMQ, ActiveMQ, JBossMQ czy Websphere MQ.
							      *
								        IceStorm - stosowany w technologii ICE
										    *
											      Specyfikacje WS-Events i WS-Notifications tu moge sie rozpisac jak trzeba

												  Dzięki dużej popularności rozwiązań typu MOM, wiele doczekało się właściwości które powodują, że są one z powodzeniem wykorzystywane w aplikacjach o duży wymaganiach - takich jak:

												      *
													        niezawodność dostarczania wiadomości zapewniana przez założenie, każda wiadomość jest autonomiczna (tj. w momencie jej wysłania rola aplikacji w przetwarzaniu danego elementu kończy się) oraz przez zastosowanie:
															          o
																	              kolejkowania i zapewnienia dostarczenia, tzw. mechanizm store-and-forward, który powoduje, że wiadomość dociera do adresata nawet jeśli dołączy on do kanału informacyjnego dopiero po jakimś czasie od wysłania wiadomości 9)
																				            o
																							            mechanizmów potwierdzeń pozwalających wysyłającemu na stwierdzenie, że wiadomość dotarła do adresata
																										    *
																											      filtrowanie wiadomości na podstawie pól nagłówka
																												      *
																													        hierarchiczność tematów mechanizmu publish/subscribe - polega na tym że wiadomości wysyłane do tematów nadrzędnych trafiają do jego wszystkich podgałęzi
																															    *
																																      mechanizmy autoryzacji wysyłania i odbierania wiadomości w oparciu o ACL z uwzględnieniem hierarchi tematów
																																	      *
																																		        obsługa transakcyjności tzn. dostarczanie wiadomości jest zablokowane do czasu, aż transakcja zostanie zakończone oraz wszystkie wiadomości zostaną pomyślnie wysłane

																																				EAI

																																				Enterprise Application Integration, to idea która pojawiła się w połowie lat 90 poprzedniego stulecia, polegająca na integracji w oparciu o jeden centralny punkt tzw. hub-and-spoke broker 10). Na punkcie tym spoczywa zadanie zawiadywania cała komunikacją w obrębie systemu - to on decyduje o tym gdzie ma trafić dana wiadomość. Dodatkowo architektura ta separuje aplikacje od właściwego kodu integrującego poprzez użycie opgrogramowania BPM (Business Process Management).

																																				korzysta z idei hub-and-spoke broker integration 11) - tj.

																																				umiejscowienie ESB wśród tych technologii pokazuje poniższy obrazek:

																																				Gdzie jest ESB i dlaczego
																																				Założenia ESB

																																				    *
																																					      rozdział z 1.7 z książki powyżej - może z obrazkami :)

																																						      *
																																							        założenia ESB + pokazać zgodność z SOA

																																									(przynajmniej tak piszą na http://en.wikipedia.org/wiki/Enterprise_service_bus) http://www-306.ibm.com/software/info1/websphere/index.jsp?tab=landings/esb

																																									    *
																																										      problemy ESB (conieco o wydajności)
																																											      *
																																												        istniejące implementacje ESB http://www-306.ibm.com/software/info1/websphere/index.jsp?tab=landings/esb
																																														    *
																																															      zastosowania ESB w prawdziwym świecie:

																																																  rozdział nr 2 książki + 1.8 ? http://www.oreillynet.com/xml/blog/2006/08/esb_adoption_in_government.html http://www.gcn.com/print/25_20/41319-1.html http://steve.vinoski.net/blog/2007/10/04/the-esb-question/
																																																  JBI ?

																																																  JBI technical overview https://open-esb.dev.java.net/kb/preview4/jbiag.html

																																																  Proponuję spojrzeć do książki rozdział 10.1 oraz do dokumentacji glassfisha

