\subsection{lip environment}
One of OOC libraries on which I want to concern is lip. \cite{kzajac} 
It was made as a utility able to help in solving Irregular \footnote{these which can't be optimized in the process of development, because their properties are not known until the runtime ( such as Monte-Carlo or N-Body simulations ). }simulations and Out of Core problems.
It is based on MPI-IO, so it can be ported on any system: from SMP machine to simple workstation.
It has three interesting features helping in more efficient use of parallel machines:

\begin{itemize}
	\item{Reusing communication schedule} - in lip we are able to construct objects called communication patterns, which are responsible for managing sending and suspending data from memory.
	We can do it in phase called ,,inspector phase'', and therefore the I/O delays are a bit minimalized ( as the data are coming while the other are being processed ).
	\item{Data caching}- which is especially used in computing OOC problems. Additionally the I/O latency is minimized by derived data attributes, that makes as able to read one non-contignous burst of data per only one MPI-IO routine. 
	\item{Scalability} - the number of processors doesn't matter ( in spite of CHAOS which needed $2^n$ ).

\end{itemize}

OOC index-arrays problems are done by dividing data for i-sections and referencing globals to the local nodes.
It's done by multiple use of Localize() method which is done in the so called Inspector phase.
Every node receives the same portion of data and processes it. 
The global chunks are treated almost like the local ones.
The only difference is in gathering data ( we have to perform request for new data ).

Sometimes problem grows a bit and also data arrays becomes OOC.
That is little more complicated as the Inspector phase ( from the first case ) is insufficient, because some data could be unreachable, as their index exists only in the disk memory.
These problems could be solved using another optimization - it should concern not only on communication but also on I/O access. 
It's kind of trick and is done by examining and translating references between disk and the memory.
Calling ooc\_localize() we can be sure that lip will make everything for us.

After such a examination and translation the computing can be done using simple MPI procedures.

\subsection{KeLP Programming system}
KeLP \cite{kelp} is a C++ library for coordinating block-structured scientific applications.
It's based on intuitive, geometric programming abstraction: points, regions, etc... It's really hard typed library ( every object type has characteristic "X" on it's end, exp. PointX and similar ).
It also uses similar inspector-executor paradigm as lip do.
The main purpose of using KeLP is parallel programming in computing - it doesn't have internal OOC support, but thanks to Bradley Broom's and Robert Fowler's work we've got a great utlity for building and converting our in-core solutions to OOC scheme.  
It's called KeLPIO \cite{kelpio} and comes with many interesting features:
\begin{itemize}
	\item{Checkpointing} - helps in overlaps saving
	\item{Embedding} - we are able to transform away overlapped regions of data
	\item{OOCArrayX} - new type which can be used in similar way as XArrayX
	\item{New way of communication} - done by using OOCMoverX
	\item{VArrayX} - new virtual interface helps in creating cleaner namespace ( which got dirter 'cause of types which meant the same~)
\end{itemize}
One disadvantage of using the KeLPIO is lack of MPI support, but it's authors promise that work is in progress.
